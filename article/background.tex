\section{Background}
% Things which should be skippable for anybody familiar with the field.
% Basically just a review of the technologies we build on and extend.
% Long commentary on methods actually goes here.
% Explain stuff like Gentoo or containers here.

Independent verification of published results is a crucial step in the establishment and maintanence of trust in shared scientific understanding.
Interest in research replication and reexecution has become more common due in part to dramatic results \cite{TODO: Salmon}, dedicated communities \cite{TODO: Cite Neurodebian}, and widely publicized large-scale reproducibility meta-analyses \cite{RPP}.
The execution of data analysis code requires a software environment, which is often non-trival and time consuming for other researchers to recreate.

\subsection{Target Article 2022, Original Execution}

In order to optimally leverage extant technologies and best-practices pertaining to reexecution in the face of high task complexity, we have selected a novel neuroimaging study, identified as OPFVTA. \cite{opfvta}.
The OPFVTA target article, which this study uses as an example, primarily maps the dopaminurgic projects of the Ventral Tegement, which is resolved via stimulus-evoked neuroimaging analysis.
The target article was designed to be fully reexecutable on Gentoo Linux with the assumption that some software updates and resultant bug fixes will be necessary.
OPFVTA analysis is executed using several high level steps, data retrieval, data processesing and registration, statistical evaluation, and a re-rendering of the final artifact with dynamically updated graphics and inline statistics.

TODO: add from data analysis


\subsection{Preserving Software Environments with Container Images}

One way of sidestepping the challenge of rebuilding a software environment is to distribute a snapshot of the entire environment, refered to as an "image".
An increasingly common modern approach to creating, distributing, and executing images makes use of the container ecosystem.
Using containers, host systems can provision guest machines that are self-contained and isolated from the host, and most importantly for our purposes, it preserves a working software environment.
"Open Container Initiative" (OCI) standard container images can be used by Docker, Podman, and Apptainer.
Singularity (recently renamed to Apptainer) has a different image format and toolset optimized for While OCI images are nearly ubiquitous in the cloud industry, Singularity (recently renamed to Apptainer) is a toolset that was developed for High Performance Computing, and has its own native specification.

\subsection{Other Methods of Preserving Software Environments}

pip-timemachine
e-life
https://github.com/datalad-handbook/repro-paper-sketch/
https://www.nature.com/articles/s42005-020-00403-4


\subsection{Preserving Versioned Collection of Components}

TODO: weak needs a lot more
A proposed set of principles, "Yodas Organigram for Data Analysis" (YODA Principles), suggests linking external repositories via git submodules. \cite{TODO YODA}
By distributing a set of separately versioned component parts, YODA "compliant" data analysis helps us to make use of modularity to improve portability, reusability, and maintainability of the research execution stack.
the basic feasibility of \textit{de novo} research output generation from the earliest recorded provenance
TODO: needs a little datalad but not too much

% \whatwewilldo-methods-results-dumping area
%
% The authors of thewill extend the portability of OPFVTA by providing a functional container snapshot of the OPFVTA execution stack at the point in time it was run to produce this meta-article.
% The authors believe \cite{help me out here?} that a preserved, functional snapshots of the software environments are useful to maintain trust in the resultant artifacts over time.
% Worse, the construction and resultant execution of computational environments is often not repeatable due to dependence on third party upstream repositories.
% %TODO @yoh, do we need to explain git?
% \whatsomeonecoulddosomeday
%
% This precludes automatic reexecution of the full research output, and limits their potential for re-use.
%
% %TODO yoh Is there a review of people sharing their code? If not we can cite a bunch of people who brag about putting their stuff on GH
% %TODO asmacdo +1 cool
%
% Due to the high task complexity of integrating all data analysis into a coherent and reliable workflow, extant efforts pertaining to full article reexecution are scant and not suitably stress-tested at a larger scale.
% One example is a novel neuroimaging study, identified as “OPFVTA” \cite{opfvta} based on author resource naming.
% The 2022 article is accompanied by a programmatic workflow via which it can be fully regenerated — based solely on raw data, data analysis instructions, and the natural-language manuscript text — and which is initiated via a simple executable script in the ubiquitous GNU Bash \cite{bash} command language.
% The reexecution process in this effort relies on an emerging infrastructure standard, RepSeP \cite{repsep}, which is used by additional other articles, thus providing a larger scope for conclusions that can be drawn from its study.

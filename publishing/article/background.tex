\section{Background}
% Things which should be skippable for anybody familiar with the field.
% Basically just a review of the technologies we build on and extend.
% Long commentary on methods actually goes here.
% Explain stuff like Gentoo or containers here.

\subsection{Reexecutable Research}

Independent verification of published results is a crucial step for establishing and maintaining trust in shared scientific understanding \cite{rpp, Ioannidis2005}.
The property of a research workflow to automatically produce an output — analogous, even if incoherent, with the original — based on the same input data and same instruction set is known as reexecutability.
This property, though conceptually simple, has remained largely unexplored as a distinct phenomenon in the broader sphere of “research reproducibility”.
The core distinction between reexecutability and reproducibility, is that the latter refers to obtaining consistent results when re-testing the same phenomenon \cite{NASrepro}, while the former refers to being able to obtain \textit{any} results, automatically, while re-using the same data and instructions.
While the scope of \textit{reexecution} is thus much narrower than that of \textit{reproduction}, it constitutes a more well-defined and therefore tractable issue in improving the quality and sustainability of research.
In all cases, reexecutability increases the feasibility of reproduction assessments, as it enables high-iteration re-testing of whatever parts of a study are automated.
Further, in the case of complex analysis processes with vast parameter spaces, reexecutability is a prerequisite for detailed reproducibility assessments.
Lastly, reexecution constitutes a capability in and of itself, with ample utility in education, training, and resource reuse for novel research purposes (colloquially, “hacking”) — which may accrue even in the absence of accurate result reproduction.

Free and Open Source Software \cite{foss} has significantly permeated the world of research, and it is presently not uncommon for researchers to publish part of their data analysis instructions under free and open licenses.
However, such analysis instructions are commonly disconnected from the research output document, which is manually constructed from static inputs.
Notably, without fully reexecutable instructions, data analysis outputs and the positive claims which they support are not verifiably linked to the methods which generate them.

Reexecutability is an emergent topic in research, with a few extant efforts attempting to provide solutions and tackle associated challenges.
Such efforts stem both from journals and independent researchers interested in the capabilities which reexecutable research processes offer to the ongoing development of their work.
Among these, an effort by the eLife journal \cite{eliferep} provides dynamic article figures based on the top-most data processing output and executable code conforming to journal standards.
NeuroLibre~\cite{neurolibre} provides a Jupyter Notebook based online platform for publishing executable books along with a selection of related assets, namely code, data, and a reexecution runtime.
Jupyter Notebooks are also used independently of journal support, yet such usage is indicative of a focus on interactivity for top-most analysis steps rather than full reexecution, and characterized by a widespread lack of either data or software dependency specification \cite{samuel2024}.
Independent researcher efforts at creating reexecution systems offer more comprehensive and flexible solutions, yet remain constrained in scope and generalizability.
For example, they may provide reference implementations which are either applied to comparatively simple analysis processes \cite{Dar2019} or tackle complex processes, but assume environment management capabilities which may not be widespread \cite{repsep}.

In order to optimally leverage extant efforts pertaining to full article reexecution and in order to test reexecutability in the face of high task complexity, we have selected a novel neuroimaging study, identified as OPFVTA (OPtogenetic Functional imaging of Ventral Tegmental Area projections) \cite{opfvta}.
The 2022 article is accompanied by a programmatic workflow via which it can be fully regenerated — based solely on raw data, data analysis instructions, and the natural-language manuscript text — and which is initiated via a simple executable script in the ubiquitous GNU Bash \cite{bash} command language.
The reexecution process in this effort relies on an emerging infrastructure approach, RepSeP \cite{repsep}, also in use by other articles, thus providing a larger scope for conclusions that can be drawn from its study.


\subsection{Data Analysis}

One of the hallmarks of scientific data analysis is its intricacy — resulting from the manifold confounds which need to be accounted for, as well as from the breadth of questions which researchers may want to address.
Data analysis can be subdivided into \emph{data preprocessing} and \emph{data evaluation}.
The former consists of data cleaning, reformatting, standardization, and sundry processes which aim to make data suitable for evaluation.
Data evaluation consists of various types of statistical modeling, commonly applied in sequence at various hierarchical steps.

The OPFVTA article, which this study uses as an example, primarily studies effective connectivity, which is resolved via stimulus-evoked neuroimaging analysis.
The stimulus-evoked paradigm is widespread across the field of neuroimaging, and thus the data analysis workflow (both in terms of \emph{data processing} and \emph{data evaluation}) provides significant analogy to numerous other studies.
The data evaluation step for this sort of study is subdivided into “level one” (i.e. within-subject) analysis, and “level two” (i.e. across-subject) analysis, with the results of the latter being further reusable for higher-level analyses \cite{Friston1995}.
In the simplest terms, these steps represent iterative applications of General Linear Modeling (GLM), at increasingly higher orders of abstraction.

Computationally, in the case of the OPFVTA article as well as the general case, the various data analysis workflow steps are sharply distinguished by their time cost.
By far the most expensive element is a substage of data preprocessing known as registration.
This process commonly relies on iterative gradient descent and can additionally require high-density sampling depending on the feature density of the data.
The second most costly step is the first-level GLM, the cost of which emerges from to the high number of voxels modeled individually for each subject and session.

The impact of these time costs on reexecution is that rapid-feedback development and debugging can be stifled if the reexecution is monolithic.
While ascertaining the effect of changes in registration instructions on the final result unavoidably necessitate the reexecution of registration and all subsequent steps — editing natural-language commentary in the article text, or adapting figure styles, should not.
To this end the reference article employs a hierarchical Bash-script structure, consisting of two steps.
The first step, consisting in data preprocessing and all data evaluation steps which operate in voxel space, is handled by one dedicated sub-script.
The second step handles document-specific element generation, i.e. inline statistics, figure, and TeX-based article generation.
The nomenclature to distinguish these two phases introduced by the authors is “low-iteration” and “high-iteration”, respectively \cite{repsep}.

Analysis dependency tracking — i.e. monitoring whether files required for the next hierarchical step have changed, and thus whether that step needs to be reexecuted — is handled for the high-iteration analysis script via the RepSeP infrastructure, but not for the low-iteration script.


\subsection{Software Dependency Management}

Beyond the hierarchically chained data dependencies, which can be considered internal to the study workflow, any data analysis workflow has additional dependencies in the form of software.
This refers to the computational tools leveraged by the workflow — which, given the diversity of research applications, may encompass numerous pieces of software.
Additionally, individual software dependencies commonly come with their own software dependencies, which may in turn have further dependencies, and so on.
The resulting network of prerequisites is known as a “dependency graph”, and its resolution is commonly handled by a package manager.

The OPFVTA article in its original form relies on Portage \cite{portage}, the package manager of the Gentoo Linux distribution.
This package manager offers integration across programming languages, source-based package installation, and wide-ranging support for neuroscience software \cite{ng}.
As such, the dependencies of the target article itself are summarized in a standardized format, which is called an ebuild — as if it were any other piece of software.
This format is analogous to the format used to specify dependencies at all further hierarchical levels in the dependency tree.
This affords a homogeneous environment for dependency resolution, as specified by the Package Manager Standard \cite{pms}.
Additionally, the reference article contextualizes its raw data resource as a dependency, integrating data provision in the same network as software provision.

While the top-level ebuild (i.e. the direct software dependency requirements of the workflow) is included in the article repository and distributed alongside it, the ebuilds which specify dependencies further down the tree are all distributed via separate repositories.
These repositories are version controlled, meaning that their state at any time point is documented, and they can thus be restored to represent the environment as it would have been generated at any point in the past.


\subsection{Software Dependencies}

The aforementioned infrastructure is relied upon to provide a full set of widely adopted neuroimaging tools, including but not limited to ANTs \cite{ants}, nipype \cite{nipype}, FSL \cite{fsl}, AFNI \cite{afni}, and nilearn \cite{nilearn}.
Nipype in particular provides workflow management tools, rendering the individual sub-steps of the data analysis process open to introspection and isolated reexecution.
Additionally, the OPFVTA study employs a higher-level workflow package, SAMRI \cite{samri,irsabi}, which provides workflows optimized for the preprocessing and evaluation of animal neuroimaging data.


\subsection{Containers}

Operating system virtualization is a process whereby an ephemeral “guest” environment is started in and may be reused between persistent “host” systems.
Virtual machines (VMs), as these “guest” environments are called, can thus provide users with environments tailored to a workflow, while eschewing the need to otherwise (e.g. manually or via a package manager) provide the tools it requires.
Once running, VMs are self-contained and isolated from the host, also eliminating the risk of unwanted persistent changes being made to the host environment.
Perhaps the most important benefit of virtual isolation is significantly improved security, allowing system administrators to safely grant users relatively unrestricted access to large-scale computational capabilities.
However, VMs can also help mitigate issues arising from package updates by locking a specific dependency resolution state which is known to work as required by a workflow, and distributing that instead of a top-level dependency specification which might resolve differently across time.

Modern advances in container technology allow the provision of the core benefits of system virtualization, but lighten the associated overhead by making limited use of the host system, specifically the hypervisor.
Container technology is widespread in industry applications, and many container images are made available via public image repositories.
While container technology has gained significant popularity specifically via the Docker toolset, it refers to an overarching effort by numerous organizations, now best represented via a Linux Foundation project, the “Open Container Initiative” (OCI).
The OCI governing body has produced an open specification for containers, which can be used by various container runtimes and toolsets.
Generally, OCI-compliant container images can be executed analogously with Docker, Podman, or other OCI compliant tools.

While OCI images are nearly ubiquitous in the software industry, Singularity (recently renamed to Apptainer) is a toolset that was developed specifically for high-performance computing (HPC) and tailored to research environments.
A significant adaptation of Singularity to HPC environments is its capability to run without root privileges.
However, recent advances in container technology have provided similar capabilities.
Further, Singularity permits the conversion of OCI images into Singularity images, and recent versions of Apptainer have also added support for natively running OCI containers — thus making reuse of images between the two technologies increasingly convenient.

Container technology thus represents a solution to providing stable reusable environments for complex processes, such as the automatic generation of research articles.
In particular, containers provide a convenient way of making advanced package management solutions — as seen in the original OPFVTA article — available to users which may lack them on their host systems.

\section{Discussion}

In this article we present an automated workflow for full, end-to-end article reexecution and quantification of results divergence at the level of the reproduced article manuscript.
We generate the full research communication output (including inline statistics, figures, and brain maps) from solely the raw data and automatically executable code.
This effort substantiates the feasibility of article reexecution as a process, based on a real-life peer-reviewed study example.
Additionally, our effort provides a sample implementation of technologies required to provide this capability and increase the adoption of reexecutable research outputs.
To this end, we also detail important and transferable principles, and document common pitfalls in creating a reexecution workflow.

\subsection{Reexecutability}
% gneneral description of the property.
We argue that reexecutability is a core aspect of reliable research output creation.
Reexecutability implies that instructions are formulated in such a way that they can be automatically deployed without human operator bias.
In contrast to arbitrary reporting standards, the property of reexecutability implicitly guarantees that instructions for all pertinent instructions are documented fully.
%Further, reexecutability is a far more scalable process, allowing the reliable estimation of the intrinsic variability of a research workflow.

We demonstrate the feasibility of full research output reexecution by integrating cutting-edge technological capabilities, and publish all resources for open access, inspection, re-use, and adaptation.
The article reexecution system which we produced isolates data and original resources, and does not make assumptions about the internal structure of a reexecutable article, and is of course, not domain-specific.
Our system addresses artcle execution via a Bash entry point, and dependency resolution via an ebuild-style specification present in the target article.
Given these two attributes, our system can conceivably be adapted to additional target articles.

We sharply distinguish between reexecutability and reproducibility.
The former refers to the capability of an analogue research output (with any or even no consistency) to be regenerated exclusively from the earliest feasible data provenance and from automatically executable instructions used to generate the original output.
The latter refers to the quality of an analogue research output (whether automatically reexecuted or manually recreated) with respect to supporting the claims of the original research output.
We further distinguish reproducibility — which may be good or poor based on the expected coherence standard — from numerically identical reproduction of statistical metrics, which we deem as a distinct quality, replicability.

\subsection{Reproducibility}

% maybe move to background under “terminology” section.
%Reexecutability makes reproduction assesments scalable a and insulated from human operator bias, in view of intrinsic workflow variability.

We supplement the reexecution workflow description of this article with a brief demonstration of how it can be used to provide a reproducibility assesment.
For this purpose we use a difference computation tool (in computational contexts known simply as “diff”) which summarizes and visually displays mismatches between a historical manuscript record and multiple reexecutions over various environments.
Such a process makes mismatches visible at-a-glance throughout the article figures and text, rendering them them easy to locate and interpret via human inspection.

Based on these results we lay out a few key findings for further reproducibility assessments.
In particular, we notice that figures which directly map output data are highly — and to a consistent extent — variable across multiple reexecution attempts.
However, in as far as such figures are accompanied by statistical evaluations, we find these to be qualitatively consistent.
This indicates that reproduction quality is not only reliant on whether or not data processing is deterministic, but also on which aspects of the top-level data the authors seek to highlight.
While the above observations describe the current article specifically, we suspect that they may reflect a phenomenon of broader relevance.

In neuroimaging workflows, the most notorious source for non-deterministic data analysis behavior is the registration.
This process commonly operates via a random starting point — specified by a seed value — and iterates according to a gradient descent algorithm.
While the toolkit used by the article reexecuted here allows the specification of a particular seed, this was not done for the Historical Manuscript Record, nor is it a feature commonly used by operators.
In light of our results the question emerges whether or not seed specification should be introduced as a best practice.
While a fixed seed would aid in numerical reproducibility, it is possible that a specific seed — whether by coincidence or \textit{ex post facto} selection — may result in anomalous conclusions.
It may then be that a stronger finding is one which is statistically robust with respect to preprocessing variability, even if this comes at the cost of compromising numerical replicability.
Conversely, it could be argued that reproduction analysis can be better targeted and more concise, if seed values were fixed to universally accepted numbers (analogous to the usage of nothing-up-my-sleeve numbers in cryptography).

\subsection{Challenges}
For this meta-article we have selected an original neuroimaging article which already published all of the instructions needed to reproduce it in its entirety from raw data and automatically executable instructions.
Even in light of this uncommon advantage, setting up a portable reexecution system has proven to be an ample effort.
Difficulties arose primarily due to the instability of the software stack.
It is common for scientific software — and more and more common as researchers become involved in software development — to be subjected to frequent interface changes, and loss of support for older dependencies.

In this article we propose container technology as a mitigation method for such fragility.
However, this is not without draw-backs, as it can make introspection more challenging.
In view of this, we defined interactive container entry points (\texttt{make} targets), whereby the user may enter the container dedicated to automatic reexecution to inspect and test changes in the environment.
Even so, on account of these containers being dedicated to automatic execution, features such as an advanced text processor are missing, and the inclusion of such features may not be ultimately desired.

A more easily surmountable challenge was data management.
Whereas the original article strove to integrate all provision of computational requirements with the package manager, the usage of containers made the cost of this one-stop solution prohibitive.
As such, Git submodules and DataLad were used, providing enhanced functionality for e.g. data version specification, but at the cost of spreading requirements provision over different technologies.

Lastly, an unavoidable challenge consisted in execution time-cost.
While not prohibitive, this time cost not only slows iterative development work, but presages a potential decrease in the feasibility of reexecution given the trend towards larger and larger data.
This means that process complexity and experimental data size may need to be evaluated in light of the diminished accessibility to such processes as reexecution.

\subsection{Outlook}

We propose a few key considerations for the further development of article reexecution — though we note that practical reuse of this system might better identify promising enhancements better than theoretical analysis.

In particular, we find that reexecutable article debugging in a container environment can be a significant challenge, and one which will only be more severe if such an environment is already implemented in the development stage of an article.
In order to provide seamless integration of both flexible development and portable reexecution, we envision a workflow system which, for each analysis step, permits either usage of locally present executables, or entry points to a container.
We implement a version of this concept for the meta-article generation, where the \texttt{make article} target which generates this article will use the local environment, and the \texttt{make container-article} executes the same code via an entry point to a \TeX{} container.

The reproduction quality assessment methods provided in this study serve as a starting point for evaluating full article reexecution, and purposefully deal with the article as a whole.
As it is the article which is the primary research output format, any other set of values which would be earmarked as outputs-of-choice for reexecution concomitantly risk overestimation and underestimation of relevant differences.
We argue that for the derivation, reuse, or reproducibility assessment of a specific article, there is no substitute for the human-readable article as the foremost output element, as it most accurately documents all variable elements in the context the statements they underpin.
Separate machine-readable summaries can at best be an approximation of the text, and barring revolutionary technological developments remain infeasible to implement at scale.
However, it should be noted that pixel-diff comparison, as showcased here, cannot provide automatic evaluation of the differences (i.e. drawing inferences on whether or not statistical thresholds have been crossed) — so machine-readable summaries certainly have advantages.
Furthermore, there are \emph{supplementary} outputs which may provide additional capabilities, not in lieu of, but in addition to the article text.
The foremost among these — specifically pertaining to neuroimaging — are top-level statistical brain maps.
Such supplementary capabilities would not only let studies generate reusable outputs (e.g. as masks or correlation analysis targets for other studies), but would also aid the inspection of the original article.
While comparison of reexecutable articles only permits the evaluation of statements made, whole-brain maps may also reveal valuable statements which may have been untapped by the original researchers.
There already exist efforts to integrate such outputs \cite{nidm}, and our workflow also produces and records all of the top-level data (statistical maps, data tables, etc.) from which the article extracts elements relevant to its statements.
Integration between the present reexecutable article system and statistical map libraries it thus a promising endeavour for further development.

Lastly, we highlight the relevance of automated workflows for reuse and adaptation.
This includes both the reexecution system published herein, and the internal workflow of the original article.
A key strength of reexecutability is that workflows can easily be derived, with a reliable starting point with respect to successful process execution.
This refers not only to reuse for novel or derived studies, but also reuse for the inspection of specific parameter modifications.
In view of this we recommend a practical approach to the work described herein, whereby the parent reexecution system repository can be considered immediately and freely available for inspection and personal exploration and re-use of the reader.

%For example, PyPI, conda and other distributions can create an exact list of versioned packages present in the environment support package reinstall using that list.
%On Debian, NeuroDebian-enhanced systems, and to a limited degree on Ubuntu systems, it is possible to use \texttt{nd\_freeze} utility from \texttt{neurodebian-freeze} package to make use of \href{https://snapshot.debian.org}{snapshot.debian.org} and \href{http://snapshot-neuro.debian.net}{snapshot-neuro.debian.net} and use frozen to specific timestamp state of the APT repositories, thus guaranteeing access to the same versions of packages.
%NeuroDocker utility~\cite{neurodocker} provides further assistance in generating container recipes and has support for \texttt{nd\_freeze} to generate containers following the best practices.



%Lastly we note the value of reexecutability for rendering a resource re-usable.
%In effect, a reexecutable workflow allows the precise inspection of specific parameter modifications, hence helping to study key decision points in workflow design.





% Mention Gentoo in the context of provenance



% chr: not really, data acquisition will always depend on external hardware... even if you think of e.g. speed tests on computers. Data aquisition is basically the interface between the digital and outside world, so the dichotomy will always remain in some shape.
% chr: also, pretty long and tangential discussion IMHO
%As a long-term outlook, we highlight that the dichotomy commonly drawn between data acquisition and data analysis (the latter being uniquely suited for reproduction) need not remain in place indefinitely.
%Data acquisition may, given a suitable context, benefit from the same sort of automation as data analysis.


%TODO 
% Analogy with a "backup" -- there is no idea if a backup is any good until it is attempted to restore ffrom the backup. The same with studies claiming to be reproducible.
% chr: it's a good analogy, but I'm unsure where to put it, or whether it adds anything to any particular section...


% TODO yoh : I do not understand this (chr) could you please expand this so I can write something more sensible about it?
%      \item Pay attention to user used for invocation of container
%
%        docker -- by default might be root, resultant files would be owned by root, not be able to move etc.
%
%        -e "UID=$(id -u)" -e "GID=$(id -g)"
%
%        and may be `-u` `-g` if used with ``docker''.
%    \end{itemize}


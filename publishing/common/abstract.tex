The value of research articles is increasingly contingent on the results of complex data analyses which substantiate their claims.
Compared to data production, data analysis more readily lends itself to a higher standard of both full transparency and repeated operator-independent execution.
This higher standard can be approached via fully reexecutable research outputs, which contain the entire instruction set for end-to-end generation of an entire article solely from the earliest feasible provenance point, in a programatically executable format.
In this study, we make use of a peer-reviewed neuroimaging article which provides complete but fragile reexecution instructions, as a starting point to formulate a new reexecution system which is both robust and portable.
We render this system modular as a core design aspect, so that reexecutable article code, data, and environment specifications could potentially be substituted or adapted.
In conjunction with this system, which forms the demonstrative product of this study, we detail the core challenges with full article reexecution and specify a number of best practices which permitted us to mitigate them.
We further show how the capabilities of our system can subsequently be used to provide reproducibility assessments, both via simple statistical metrics and by visually highlighting divergent elements for human inspection.
We argue that fully reexecutable articles are thus a feasible best practice, which can greatly enhance the understanding of data analysis variability and the trust in results.
Lastly, we comment at length on the outlook for reexecutable research outputs and encourage re-use and derivation of the system produced herein.
% Maybe also mention, though ideally somewhere else:
% * The comparison article we thus produce, i.e. this article, is similarly reexecutable.
% * ... a collection of best practices, including YODA principles for data management, containers for preservation, Gentoo for long term flexibility.
% * This is a very good point, though again, maybe not for the abstract “The inability to re-use code due to instability thus limits the lasting value of the work as a repository of procedural knowledge”.
% *We document a number of prominent difficulties with de novo article generation, arising from the rapid evolution of extrinsic tools, and from nondeterministic data analysis procedures.


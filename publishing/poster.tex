\usepackage[british]{babel} % decent hyphenation, avoiding e.g. anal-ysis
\usepackage[iso]{isodate}
\usepackage{sansmath}
\usepackage{booktabs}
\usepackage{graphicx}
%\usepackage{sty/graphviz}
\usepackage{makecell}
%\usepackage{minted}
\usepackage{multicol}
\usepackage{siunitx}
\usepackage{nth}
\usepackage{subcaption}
\usepackage[section]{placeins}
\usepackage{pdfpages}
\usepackage{tcolorbox}


\title{
	\href{https://github.com/con/opfvta-reexecution}{
		\Large github.com/con/opfvta-reexecution
	}\\\vspace{.15em}
	Neuroimaging Article Reexecution and Reproduction Assesment System
}
\author{
	Horea-Ioan Ioanas$^{1}$,
	Austin Macdonald$^{1}$,
	Yaroslav O. Halchenko$^{1}$
}
\institute[CON]{$^{1}$Center for Open Neuroscience, Department of Psychological and Brain Sciences, Dartmouth College}
\date{\today}

\newlength{\columnheight}
\setlength{\columnheight}{0.881\textheight}

\begin{document}

\begin{frame}
\begin{columns}
	\begin{column}{.42\textwidth}
		\begin{beamercolorbox}[center]{postercolumn}
			\begin{minipage}{.98\textwidth}  % tweaks the width, makes a new \textwidth
				\parbox[t][\columnheight]{\textwidth}{ % must be some better way to set the the height, width and textwidth simultaneously
					\begin{myblock}{Abstract}
						\section{Abstract}

As complexity increases, the value of many research articles is inextricably contingent on data analysis results which substantiate their claims.
Unlike data production steps, data analysis steps lend themselves to a higher standard of both transparency and repeated operator-free execution.
This is accomplished via fully reexecutable research outputs, which contain the entire instruction set for end-to-end generation of an entire article solely from the earliest feasible provenance point, in a programatically executable format.
In this study, we make use of a peer-reviewed neuroimaging article, which provides complete but fragile reexecution instructions, in order to formulate a new reexecution model which is both robust and portable.
We render this model modular as a core design aspect, so that reexecutable article code, data, and environment specifications can be easily substituted or adapted.
In conjunction with this model, which forms the main reusable product of this study, we detail the core challenges with full article reexecution, and specify a number of best practices which permitted us to mitigate them.
We further show how these capabilities can subsequently be used in order to provide reproducibility assesments, both via simple statistical metrics, and by visually highlighting divergent elements for human inspection.
We argue that reexecutable articles are thus a feasible best practice, the usage of which can greatly enhance the understanding of data analysis variability.
Lastly, we comment at length on the outlook for reexecutable resource and encourage re-use and derivation of the model produced herein, 

% Maybe also mention, though ideally somewhere else:
% * The comparison article we thus produce, i.e. this article, is similarly reexecutable.
% * ... a collection of best practices, including YODA principles for data management, containers for preservation, Gentoo for long term flexibility.
% * This is a very good point, though again, maybe not for the abstract “The inability to re-use code due to instability thus limits the lasting value of the work as a repository of procedural knowledge”.
% *We document a number of prominent difficulties with de novo article generation, arising from the rapid evolution of extrinsic tools, and from nondeterministic data analysis procedures.


					\end{myblock}\vfill
					\vspace{-0.3em}
					\begin{myblock}{Workflow}
						\vspace{0.5em}
						\begin{figure}
							\includegraphics[width=0.99\textwidth]{figs/workflow.pdf}
							\caption{
								The reexecution system encompasses both the original article (first target), and the “meta-article” publishing materials (article manuscript, as well as this poster), the latter of which takes user- and developer-submitted reexecution results as an input for the reproduction quality assessment. 
							}
						\end{figure}
					\end{myblock}\vfill
					\vspace{-0.3em}
					\begin{myblock}{Topoplogy}
						\vspace{0.5em}
						\begin{figure}
							\captionsetup{width=.9\linewidth}
							\includegraphics[width=0.99\textwidth]{figs/topology.pdf}
							\caption{
								The reexecution workflow is supported by a resource topology in which reexecution code (first box), “meta-article” code (second box), reexecution resources (third box), and the reexecution output record (last box) are separated at the top directory level.
								The figure depicts direcotry trees via nested boxes, with external resources automatically fetched as via the reexecution code being highlighted in orange.
								The green highlighted article represents a sample reexecution output, and the blue hignlighted article represents the manuscript, an analogous output to this poster generated in the same directory.
							}
						\end{figure}
					\end{myblock}\vfill
					\begin{myblock}{Best Practice Guidelines}
						\vspace{0.5em}
						\Large
						\begin{itemize}
							\item Errors should be fatal more often than not
							\item Avoid assuming or hard-coding absolute paths to resources
							\item Avoid assuming a directory context for execution
							\item Workflow granularity greatly benefits efficienc
							\item Container image size should be kept small
							\item Resources should be bundled into a DataLad superdataset
							\item Containers should fit the scope of the underlying workflow steps
							\item Do not write debug-relevant data inside the container
							\item Parameterize scratch directories
							\item Dependency versions inside container environments should be frozen as soon as feasible
						\end{itemize}
					\end{myblock}\vfill
		}\end{minipage}\end{beamercolorbox}
	\end{column}
	\begin{column}{.59\textwidth}
		\begin{beamercolorbox}[center]{postercolumn}
			\begin{minipage}{.98\textwidth} % tweaks the width, makes a new \textwidth
				\parbox[t][\columnheight]{\textwidth}{ % must be some better way to set the the height, width and textwidth simultaneously
					\begin{myblock}{Reproduction Assessment Showcase}
						\vspace{-0.45em}
						\begin{minipage}{.56\textwidth}
						\begin{figure}
							\includegraphics[width=0.95\textwidth]{figs/diff_pages.pdf}
							\vspace{0.2em}
							\caption{
								Page-wise voxel difference comparison across multiple reexecutions in different environments indicates consistency of variability in both extent and location.
							}
							\label{fig:ras_s}
						\end{figure}
						\begin{figure}
							\fbox{\includegraphics[width=0.9\textwidth]{figs/diff_fig.pdf}}
							\vspace{0.2em}
							\caption{
								One notable source of variability are data plots, where it can be observed that even as data points vary to an almost full extent, statistical summaries can remain constant.
							}
						\end{figure}
						\end{minipage}
						\hfill
						\begin{minipage}{.38\textwidth}
						\vspace{1.1em}
						\begin{figure}
							\fbox{\includegraphics[width=0.9\textwidth]{figs/diff_text.pdf}}
							\vspace{0.2em}
							\caption{
								Text differences in statistical summaries account for a small proportion of voxel differences, but can remain well-localized instead of spreading via test shift if statistical summaries are appropriately trimmed down to a constant length.
							}
						\end{figure}
						\begin{figure}
							\fbox{\includegraphics[width=0.9\textwidth]{figs/diff_date.pdf}}
							\vspace{0.2em}
							\caption{
								A good litmus test for monitoring differences (accounting for the baseline difference in \cref{fig:ras_s}) is the datestamp of the reexecution, which should always be expected to differ from the manuscript.
							}
						\end{figure}
						\end{minipage}
					\end{myblock}\vfill
					\begin{myblock}{Full Document comparison}
						\vspace{0.75em}
							%\includepdf[pages=-,frame,scale=.1,pagecommand={}]{data/paperdiff_singularity_20230908122618.pdf}
						\fbox{\includegraphics[page=4,scale=1.04]{data/paperdiff_singularity_20230908122618.pdf}}
						\fbox{\includegraphics[page=5,scale=1.04]{data/paperdiff_singularity_20230908122618.pdf}}
						\\
						\vspace{.3em}
						\fbox{\includegraphics[page=13,scale=1.04]{data/paperdiff_singularity_20230908122618.pdf}}
						\fbox{\includegraphics[page=14,scale=1.04]{data/paperdiff_singularity_20230908122618.pdf}}
					\end{myblock}\vfill
		}\end{minipage}\end{beamercolorbox}
	\end{column}
\end{columns}
\end{frame}
\end{document}

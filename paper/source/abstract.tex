\section{Abstract}

The value of experimental research articles is inextricably contingent on data analysis results, which substantiate their natural language claims.
Such processes are highly and increasingly complex, and the instructions required to generate them are a cornerstone element of scientific work.
However, their intricacy and high reliance on extraneous tools makes these analysis processes potentially fragile.
Thus it is of crucial importance for full article generation instructions to be not only recorded and accessible, but also formatted in such a way as to support robust reexecution.
In this article we examine a neuroimaging study which already publishes full article reexecution instructions, and detail the most prominent difficulties, arising chiefly from API stability of dependency libraries.
Further, we propose an emerging reference reexecution standard which leverages explicit dependency management as well as container technologies, in order to improve portability, provenance tracking, and ease of reexecution.
Lastly, we ascertain the reproduction accuracy within the new environment in light of common non-deterministic steps in data processing, and discuss how to reduce the most widespread instances of non-deterministic behaviour.


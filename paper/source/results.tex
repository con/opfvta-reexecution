\section{Results}
% What we accomplihesd, everything which we created, invented, etc. should preferentially go here.
% Descriptive, opinions either absent, or as objective-ish comments when facts are stated.
% Should, could, might, etc. sparingly and ideally not here but in discussion.

\section{Reproduction}

We compare the difference between the Historical Manuscript Record — a product of the original executable article generation — and the Novel Workflow Reexecution, and find strong overall coherence at the qualitative but not at the quantitative level.
This coherence manifests in the statements from the original article remaining valid with regard to statistical summaries which emerge from  \textit{de novo} data processing (as seen in \ref{fig:diff_text}, \ref{fig:diff_fig}).
Quantitative incoherence is in the form of numerical variability of automatically generated metrics in the article text (\ref{fig:diff_text}), variability of sample distributions (\ref{fig:diff_fig}), as well as expected and predictable metadata changes (\ref{fig:diff_date}).


\begin{figure}
	\centering
	\includegraphics[clip,width=0.99\textwidth]{figs/paperdiff_pages}
	\caption{
		\textbf{Page-wise visual differences between the Historical Manuscript Record and the Novel Workflow Reexecution help identify overall reproduction fidelity, and identify pages with noteworthy differences.}
		Depicted are rasterized document differences, weighted 1 for changes in any pixel color channel, and rounded to four decimal points.
		The scale is logarithmic, and the hue color map is linear.
	}
	\label{fig:paperdiff_pages}
\end{figure}

\begin{figure}
	\centering
	\begin{subfigure}{0.49\textwidth}
		\centering
		\tcbox{
			\includegraphics[width=0.48\textwidth]{figs/diff_date}
			}
		\caption{
			The date change is correctly identified throughout the document.
		}
		\label{fig:diff_date}
	\end{subfigure}
	\hfill
	\begin{subfigure}{0.49\textwidth}
		\centering
		\tcbox{
			\includegraphics[width=0.48\textwidth]{figs/diff_text}
			}
		\caption{
			Statistical summary values change, but maintain qualitative evaluation bracket with respect to e.g. p-value thresholds.
		}
		\label{fig:diff_text}
	\end{subfigure}
	\\
	\vspace{1em}
	\begin{subfigure}{0.99\textwidth}
		\centering
		\tcbox{
			\includegraphics[width=0.8\textwidth]{figs/diff_fig}
			}
		\caption{
			In regression analysis, data points are highly variable, yet maintain a consistent slope and significance.
		}
		\label{fig:diff_fig}
	\end{subfigure}
	\caption{
		\textbf{The article difference showcases expected quantitative and metadata variability, while maintaining overall validity of qualitative statements.}
		The figures are extracted from a full article \texttt{diff}, with hue-shifted highlighting (red for the Historical Manuscript Record, and blue for the Novel Workflow Reexecution).
	}
	\label{fig:diff}
\end{figure}


\subsection{Repository Structure}
%TODO cite YODA with a DOI? It's covered in the handbook but apparently no DOI or ISBN... If not we can make a results section on yoda here and become the DOI source.
In order to improve the reexecution reliability of the OPVFTA article we have constructed a parent repository encompassing the original article (including updates resulting as part of this study), the raw data it operates on, as well as container image and container execution instructions.
This layout type is known as YODA (a recursive acronym for “YODAs Organigram on Data Analysis”) and aims to facilitate reproducible research and good provenance tracking practices.

%TODO make dotfile to show the layout, with a human-language cation describing where stuff is.

\begin{figure}
	\centering
	\includegraphics[clip,width=0.99\textwidth]{figs/topology}
	\caption{
		This is a caption
	}
	\label{fig:topology}
\end{figure}

Notably, this repository structure diverges from the original reference article in directly linking the data at the repository level, as opposed to relying on its installation via the package manager.
Within the context of container usage, this affords the advantage of keeping the data packages separate and not adding their disk space requirements to those of the container image.

In this novel repository structure, external components of the article are managed as Git submodules, with data sumbodules being handled by git-annex, a versioning technology suitable for binary files, and made accessible via datalad. 

